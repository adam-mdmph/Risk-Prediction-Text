% Options for packages loaded elsewhere
\PassOptionsToPackage{unicode}{hyperref}
\PassOptionsToPackage{hyphens}{url}
%
\documentclass[
]{book}
\title{Manual for Risk Prediction in Vascular Surgery}
\author{Adam Johnson}
\date{2022-01-08}

\usepackage{amsmath,amssymb}
\usepackage{lmodern}
\usepackage{iftex}
\ifPDFTeX
  \usepackage[T1]{fontenc}
  \usepackage[utf8]{inputenc}
  \usepackage{textcomp} % provide euro and other symbols
\else % if luatex or xetex
  \usepackage{unicode-math}
  \defaultfontfeatures{Scale=MatchLowercase}
  \defaultfontfeatures[\rmfamily]{Ligatures=TeX,Scale=1}
\fi
% Use upquote if available, for straight quotes in verbatim environments
\IfFileExists{upquote.sty}{\usepackage{upquote}}{}
\IfFileExists{microtype.sty}{% use microtype if available
  \usepackage[]{microtype}
  \UseMicrotypeSet[protrusion]{basicmath} % disable protrusion for tt fonts
}{}
\makeatletter
\@ifundefined{KOMAClassName}{% if non-KOMA class
  \IfFileExists{parskip.sty}{%
    \usepackage{parskip}
  }{% else
    \setlength{\parindent}{0pt}
    \setlength{\parskip}{6pt plus 2pt minus 1pt}}
}{% if KOMA class
  \KOMAoptions{parskip=half}}
\makeatother
\usepackage{xcolor}
\IfFileExists{xurl.sty}{\usepackage{xurl}}{} % add URL line breaks if available
\IfFileExists{bookmark.sty}{\usepackage{bookmark}}{\usepackage{hyperref}}
\hypersetup{
  pdftitle={Manual for Risk Prediction in Vascular Surgery},
  pdfauthor={Adam Johnson},
  hidelinks,
  pdfcreator={LaTeX via pandoc}}
\urlstyle{same} % disable monospaced font for URLs
\usepackage{longtable,booktabs,array}
\usepackage{calc} % for calculating minipage widths
% Correct order of tables after \paragraph or \subparagraph
\usepackage{etoolbox}
\makeatletter
\patchcmd\longtable{\par}{\if@noskipsec\mbox{}\fi\par}{}{}
\makeatother
% Allow footnotes in longtable head/foot
\IfFileExists{footnotehyper.sty}{\usepackage{footnotehyper}}{\usepackage{footnote}}
\makesavenoteenv{longtable}
\usepackage{graphicx}
\makeatletter
\def\maxwidth{\ifdim\Gin@nat@width>\linewidth\linewidth\else\Gin@nat@width\fi}
\def\maxheight{\ifdim\Gin@nat@height>\textheight\textheight\else\Gin@nat@height\fi}
\makeatother
% Scale images if necessary, so that they will not overflow the page
% margins by default, and it is still possible to overwrite the defaults
% using explicit options in \includegraphics[width, height, ...]{}
\setkeys{Gin}{width=\maxwidth,height=\maxheight,keepaspectratio}
% Set default figure placement to htbp
\makeatletter
\def\fps@figure{htbp}
\makeatother
\setlength{\emergencystretch}{3em} % prevent overfull lines
\providecommand{\tightlist}{%
  \setlength{\itemsep}{0pt}\setlength{\parskip}{0pt}}
\setcounter{secnumdepth}{5}
\usepackage{booktabs}
\ifLuaTeX
  \usepackage{selnolig}  % disable illegal ligatures
\fi
\usepackage[]{natbib}
\bibliographystyle{plainnat}

\begin{document}
\maketitle

{
\setcounter{tocdepth}{1}
\tableofcontents
}
\hypertarget{about}{%
\chapter{About}\label{about}}

This book is intended as an explanatory text to support the risk tool available at www.vascalc.org.

\hypertarget{usage}{%
\section{Usage}\label{usage}}

Each chapter here reflects a page of the risk tool that should outline a certain clinical scenario that requires a decision to be made. Each chapter will be broken down into sections that will outline the decision making and available evidence used to provide each presented risk prediction. This risk calculator is developed for the sole purpose of decision support and resuls should not superscede clinician and patient preference.

\hypertarget{methods}{%
\section{Methods}\label{methods}}

The initial algorithms included in this project were found through a literature search of available medical databases, bibliography reviews and referrals from content experts. Published algorithms were reviewed and included if they met the following inclusion criteria.

\begin{enumerate}
\def\labelenumi{\arabic{enumi}.}
\tightlist
\item
  Input variables available in the pre-operative setting.
\item
  Outcome variable relevant to decision making.
\item
  Full regression model with beta coefficients and intercept publically available in publication or through contact with publication authors.
\end{enumerate}

Available risk models were then reviewed and included based on their quality. Quality of risk model was determined through.

\begin{enumerate}
\def\labelenumi{\arabic{enumi}.}
\tightlist
\item
  Accuracy assessments, such as AUC, sensitivity or specificity assessments.
\item
  Parsimonious input variable selection and clear description of variable manipulation.
\item
  Homogenous patient population that aligns with the clinical question.
\item
  Transparent stakeholder engagement and algorithm development.
\end{enumerate}

\hypertarget{feedback}{%
\section{Feedback}\label{feedback}}

For suggestions, comments or questions please submit an issue on our \href{https://github.com/adam-mdmph/vascalc/issues}{github page} or \href{mailto:vascularcalculator@gmail.com}{send us an email}.

\hypertarget{aaa}{%
\chapter{Abdominal Aortic Aneurysms (AAA)}\label{aaa}}

The aim of this risk calculator is to assist in the management of patients with asymptomatic infrarenal aortic aneurysms found through screening or incidentally.

\hypertarget{input-variables}{%
\section{Input variables}\label{input-variables}}

\hypertarget{age}{%
\subsection{Age}\label{age}}

This describes the expected age of the patient at the time of the procedure. This is likely the same age as the patient at the time of the evaluation.

\hypertarget{sex}{%
\subsection{Sex}\label{sex}}

\hypertarget{race}{%
\subsection{Race}\label{race}}

\hypertarget{d-procedural-mortality}{%
\section{30d Procedural Mortality}\label{d-procedural-mortality}}

The model used for this risk prediction comes from the VSGNE published in 2015. \citet{eslami2015}

\hypertarget{post-operative-myocardial-infarcation}{%
\section{Post Operative Myocardial Infarcation}\label{post-operative-myocardial-infarcation}}

\hypertarget{clti}{%
\chapter{CLTI}\label{clti}}

The aim of this risk calculator is to assist in the management of patients with asymptomatic infrarenal aortica aneurysms found through screening or incidentally.

\hypertarget{input-variables-1}{%
\section{Input variables}\label{input-variables-1}}

\hypertarget{post-procedural-mortality}{%
\section{Post-procedural mortality}\label{post-procedural-mortality}}

\hypertarget{post-operative-myocardial-infarction}{%
\section{Post operative myocardial infarction}\label{post-operative-myocardial-infarction}}

\hypertarget{carotid}{%
\chapter{Carotid}\label{carotid}}

The aim of this risk calculator is to assist in the management of patients presenting with carotid artery stenosis and determining the best management strategy.

\hypertarget{input-variables-2}{%
\section{Input Variables}\label{input-variables-2}}

\hypertarget{year-stroke-risk}{%
\section{5 year stroke risk}\label{year-stroke-risk}}

\hypertarget{post-operative-myocardial-infarction-1}{%
\section{Post operative Myocardial Infarction}\label{post-operative-myocardial-infarction-1}}

  \bibliography{references.bib}

\end{document}
